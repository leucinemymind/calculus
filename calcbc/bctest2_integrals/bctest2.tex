\documentclass[letterpaper, 12pt]{article}
\usepackage{graphicx} % Required for inserting images
\usepackage{textcomp}
\usepackage{fullpage}
\usepackage{amsmath}
\usepackage{xcolor}
\usepackage{float}
\usepackage{geometry}
\geometry{margin=1in}
\usepackage{enumitem}
\usepackage{hyperref}
\usepackage{microtype}
\usepackage{gensymb}
\usepackage{parskip}
\usepackage{tikz}
\usepackage{pgfplots}
\pgfplotsset{compat=1.18}
\usepackage{nicefrac}
\hypersetup{
    colorlinks=true,        % Enable colored links
    linkcolor=teal,         % Set color for internal links
    citecolor=teal,         % Set color for citations
    filecolor=teal,         % Set color for file links
    urlcolor=teal           % Set color for URLs
}

\newcommand{\example}[1]{\textcolor{blue}{\textbf{Example:} #1}}
\newcommand{\step}[2]{\textcolor{blue}{\textbf{Step #1:} #2}}

\begin{document}

\begin{center}
\textbf{{\Large AP Calculus BC Test 2}}
\end{center}

\subsection*{H10 Linearization}

Linearization is a method used to approximate the value of a function near a given point using the tangent line at that point. 

\subsection*{H11 Optimization}

\subsection*{H12 Estimating with Riemann sums}

\subsection*{H13 Writing and interpreting Riemann sums}

\subsection*{H14 Riemann sums and definite integrals}

\subsection*{H15 Antiderivatives and indefinite integration}

\subsection*{H16 The Fundamental Theorem of Calculus}

\subsection*{H17 Integration by u-substitution and change of variable}

\subsection*{H18 Inverse trig integration}

\subsection*{H19 Long division and integration}

\subsection*{How to know which integration method to use}

\end{document}