\documentclass[letterpaper, 12pt]{article}
\usepackage{graphicx} % Required for inserting images
\usepackage{textcomp}
\usepackage{fullpage}
\usepackage{amsmath}
\usepackage{xcolor}
\usepackage{float}
\usepackage{enumitem}
\usepackage{geometry}
\geometry{margin=1in}
\usepackage{enumitem}
\usepackage{hyperref}
\usepackage{microtype}
\usepackage{gensymb}
\usepackage{parskip}
\usepackage{tikz}
\usepackage{pgfplots}
\pgfplotsset{compat=1.18}
\usepackage{nicefrac}
\hypersetup{
    colorlinks=true,        % Enable colored links
    linkcolor=teal,         % Set color for internal links
    citecolor=teal,         % Set color for citations
    filecolor=teal,         % Set color for file links
    urlcolor=teal           % Set color for URLs
}

\DeclareMathOperator{\arcsec}{arcsec}

\newcommand{\example}[1]{\textcolor{blue}{\textbf{Example:} #1}}
\newcommand{\step}[2]{\textcolor{blue}{\textbf{Step #1:} #2}}

\begin{document}

\begin{center}
\textbf{{\Large AP Calculus BC Test 2}}
\end{center}

\subsection*{H10 Linearization}

Linearization is a method used to approximate the value of a function near a given point using the tangent line at that point. 

\example{Approximate the cube root of 82 using linearization.}

\step{1}{Identify the function and the point of tangency.}

$$f(x) = \sqrt[3]{x}, \quad a = 81$$

\step{2}{Find the derivative of the function.}

$$f'(x) = \frac{1}{3x^{2/3}}$$

\step{3}{Evaluate the function and its derivative at the point of tangency.}

$$f(81) = 3, \quad f'(81) = \frac{1}{27}$$

\step{4}{Write the equation of the tangent line.}

\subsection*{H11 Optimization}

Optimization involves finding the maximum or minimum values of a function subject to certain constraints.

\example{A farmer wants to build a rectangular pen with a fixed amount of fencing (100 meters). What dimensions will maximize the area of the pen?}

\step{1}{Define the variables and equations.}

Let $x$ be the length and $y$ be the width of the pen. The area $A$ is given by

$$ A = xy $$

and the constraint from the fencing is

$$ 2x + 2y = 100 $$

or

$$ x + y = 50 $$

\step{2}{Express one variable in terms of the other using the constraint.}

$$ y = 50 - x $$

\step{3}{Substitute into the area equation.}

\begin{gather*}
A(x) = x(50 - x) \\
= 50x - x^2
\end{gather*}

\step{4}{Find the critical points by taking the derivative and setting it to zero.}

\begin{gather*}
A'(x) = 50 - 2x \\
0 = 50 - 2x
-2x = -50
x = 25
\end{gather*}

\step{5}{Find the corresponding value of $y$.}

$$y = 50 - 25 = 25$$

\step{6}{Conclude with a justification:}

The dimensions that maximize the area of the pen are 25 meters by 25 meters because the sign of the first derivative changes from positive to negative at $x = 25$, indicating a maximum.

\subsection*{H12 Estimating with Riemann sums}

\begin{description}
\item [RRAM] Right Rectangular Approximation Method - heights of rects = heights of right endpoints
\item [LRAM] Left Rectangular Approximation Method - heights of rects = heights of left endpoints
\item [MRAM] Midpoint Rectangular Approximation Method - heights of rects = heights of midpoints
\item [Trapezoidal approximation] Average of RRAM and LRAM
\end{description}

\example{Estimate with MRAM the value of $\displaystyle \int_{0}^{4} (x^2 + 1) \: dx$ using 4 subintervals.}

\step{1}{Determine $\Delta x$.}

$$\Delta x = \frac{4 - 0}{4} = 1$$

\step{2}{Identify the midpoints of each subinterval.}

\begin{itemize}
\item Subinterval $[0, 1]$: midpoint = $0.5$
\item Subinterval $[1, 2]$: midpoint = $1.5$
\item Subinterval $[2, 3]$: midpoint = $2.5$
\item Subinterval $[3, 4]$: midpoint = $3.5$
\end{itemize}

\step{3}{Evaluate the function at each midpoint.}

\begin{gather*}
f(0.5) = (0.5)^2 + 1 = 1.25 \\
f(1.5) = (1.5)^2 + 1 = 3.25 \\
f(2.5) = (2.5)^2 + 1 = 7.25 \\
f(3.5) = (3.5)^2 + 1 = 13.25
\end{gather*}

\step{4}{Calculate the Riemann sum.}

\begin{gather*}
\text{MRAM} = \Delta x [f(0.5) + f(1.5) + f(2.5) + f(3.5)] \\
= 1 [1.25 + 3.25 + 7.25 + 13.25] \\
= \boxed{25}
\end{gather*}

\subsection*{H13 Writing and interpreting Riemann sums}

When converting a Riemann sum of the form

$$\lim_{n \to \infty} \sum_{i=1}^{n} \frac{b}{n} f\left(a + \frac{bi}{n}\right)$$

to a definite integral, follow these steps:

\begin{enumerate}
\item Identify $\Delta x$ (the width of each subinterval) as $\frac{b}{n}$.
\item Identify the function $f(x)$ inside the sum.
\item Determine the limits of integration:
    \begin{itemize}
    \item The lower limit is $a$ (the starting point of the interval).
    \item The upper limit is $a + b$ (the endpoint of the interval).
    \end{itemize}
\item Write the definite integral as

$$\int_{a}^{a+b} f(x) \: dx$$
\end{enumerate}

Similarly, when writing a Riemann sum for a definite integral of the form

$$\int_{a}^{a+b} f(x) \: dx$$

follow these steps:

\begin{enumerate}
\item Identify the interval $[a, a+b]$ and the function $f(x)$.
\item Determine $\Delta x$ as $\frac{b}{n}$ (subtract upper limit of integration from lower) and plug into the equation for $x$ as $\frac{bi}{n}$
\item Write the Riemann sum as

$$\lim_{n \to \infty} \sum_{i=1}^{n} \frac{b}{n} f\left(a + \frac{bi}{n}\right)$$

\end{enumerate}

\example{Write the Riemann sum for the definite integral $\displaystyle \int_{2}^{5} (x^2 + 1) \: dx$.}

\step{1}{Identify the interval and function.}

The interval is $[2, 5]$ and the function is $f(x) = x^2 + 1$.

\step{2}{Determine $\Delta x$ and express $x_i$.}

\begin{gather*}
\Delta x = \frac{5 - 2}{n} = \frac{3}{n} \\
x_i = 2 + \frac{3i}{n}
\end{gather*}

\step{3}{Write the Riemann sum.}

$$\boxed{\lim_{n \to \infty} \sum_{i=1}^{n} \frac{3}{n} \left[ \left(2 + \frac{3i}{n}\right)^2 + 1 \right]}$$

\subsection*{H15 Definite Integration}

\subsubsection*{Properties of integration}

\begin{gather*}
\text{Flipping the limits:} \quad \int_{a}^{b} f(x) \: dx = -\int_{b}^{a} f(x) \: dx \\
\text{Sum/Difference:} \quad \int_{a}^{b} [f(x) \pm g(x)] \: dx = \int_{a}^{b} f(x) \: dx \pm \int_{a}^{b} g(x) \: dx \\
\text{Constant multiple:} \quad \int_{a}^{b} c f(x) \: dx = c \int_{a}^{b} f(x) \: dx \\
\text{Additivity over intervals:} \quad \int_{a}^{b} f(x) \: dx + \int_{b}^{c} f(x) \: dx = \int_{a}^{c} f(x) \: dx \\
\text{Even function:} \quad \int_{-a}^{a} f(x) \: dx = 2 \int_{0}^{a} f(x) \: dx \\
\text{Odd function:} \quad \int_{-a}^{a} f(x) \: dx = 0 \\
\text{Integral of a constant:} \quad \int_{a}^{b} c \: dx = c(b - a)
\end{gather*}

\example{$\displaystyle \int_0^\pi \sin x \: dx = 2$. Given this integral, find:}

\begin{enumerate}[label=\alph*)]
\item $\displaystyle \int_{\pi}^{2\pi} \sin x \, dx$
\item $\displaystyle \int_0^{2\pi} 3 \sin x \, dx$
\item $\displaystyle \int_0^{\frac{\pi}{2}} 3 \sin x \, dx$
\item $\displaystyle \int_0^{\pi} 3 \sin x \, dx$
\item $\displaystyle \int_{-\pi}^{\pi} \sin x \, dx$
\end{enumerate}

Answers:

\begin{enumerate}[label=\alph*)]
\item $-2$ (flipped limits)
\item $6$ (constant multiple)
\item $3$ (additivity over intervals)
\item $6$ (constant multiple)
\item $0$ (odd function)
\end{enumerate}


\subsection*{H15 Antiderivatives and indefinite integration}

Antiderivative rules are the same as derivative rules, but in reverse. Always +C to the end of indefinite integrals.

\example{Find the antiderivative of $\displaystyle f(x) = 3x^2 - 4x + 5$.}

Apply the power rule to each term:

\begin{gather*}
\int (3x^2 - 4x + 5) \: dx \\
= \int (\frac{3x^{2+1}}{2+1} - \frac{4x^{1+1}}{1+1} + \frac{5x^{0+1}}{0+1}) \: dx \\
= \boxed{x^3 - 2x^2 + 5x + C}
\end{gather*}

\subsection*{H16 The Fundamental Theorem of Calculus}

\subsubsection*{First Fundamental Theorem}

If $f$ is continuous on $[a, b]$ and $F$ is an antiderivative of $f$ on $[a, b]$, then

$$\int_{a}^{b} f(x) \: dx = F(b) - F(a)$$

\subsubsection*{Second Fundamental Theorem}

The derivative of the integral of a function is the original function:

$$\frac{d}{dx} \int_{a}^{x} f(t) \: dt = f(x)$$

\example{Evaluate $\displaystyle \frac{d}{dx} \int_{1}^{3x} (2t + 3) \: dt$.}

\step{1}{Use the Second FTC and plug in the upper limit ($3x$) into the function.}

\begin{gather*}
2(3x)+3 \\
= 6x + 3
\end{gather*}

\step{2}{Multiply by the derivative of the upper limit.}

\begin{gather*}
(6x + 3) \cdot \frac{d}{dx}(3x) \\
= (6x + 3) \cdot 3 \\
= \boxed{18x + 9}
\end{gather*}

\subsubsection*{Average value}

The \textbf{average value} of a function $f$ on the interval $[a, b]$ is given by

$$f_{\text{avg}} = \frac{1}{b - a} \int_{a}^{b} f(x) \: dx$$.

\subsection*{H17 Integration by $u$-substitution and change of variable}

Use $u$-substitution when the integral contains a function and its derivative.

\example{Evaluate $\displaystyle \int 2x \sqrt{x^2 + 1} \: dx$.}

\step{1}{Choose $u$ to be the inner function.}

\begin{gather*}
u = x^2 + 1 \\
du = 2x \: dx
\end{gather*}

\step{2}{Rewrite the integral in terms of $u$.}

$$\int \sqrt{u} \: du$$

\step{3}{Integrate with respect to $u$.}

\begin{gather*}
\int \sqrt{u} \: du = \int u^{1/2} \: du \\
= \frac{u^{3/2}}{(3/2)} + C \\
= \frac{2}{3} u^{3/2} + C
\end{gather*}

\step{4}{Substitute back to $x$.}
$$\boxed{\frac{2}{3} (x^2 + 1)^{3/2} + C}$$

Sometimes, $u$-substitution requires a \textbf{change of variable} when the integral does not directly contain the derivative of the inner function.

\example{Evaluate $\displaystyle \int x\sqrt{x-1} \: dx$.}

\step{1}{Choose $u$ to be the inner function.}

\begin{gather*}
u = x - 1 \\
du = dx \\
\end{gather*}

This $u$-substitution is not sufficient because there is still an $x$ in the integral. To fix this, apply a change of variable.

\step{2}{Solve for $x$ in terms of $u$.}

$$ x = u + 1 $$

\step{3}{Substitute.}

\begin{gather*}
\int x\sqrt{x-1} \: dx \\
= \int (u + 1) \sqrt{u} \: du \\
= \int (u + 1) u^{1/2} \: du \\
\end{gather*}

\step{4}{Distribute.}

\begin{gather*}
\int (u+1) u^{1/2} \: du \\
= \int (u^{3/2} + u^{1/2}) \: du
\end{gather*}

\step{5}{Integrate with respect to $u$.}

\begin{gather*}
\int (u^{3/2} + u^{1/2}) \: du \\
= \boxed{\frac{2}{5} u^{5/2} + \frac{2}{3} u^{3/2} + C}
\end{gather*}

For definite integrals, either substitute back to $x$ or change the limits of integration to $u$.

\example{Evaluate the same integral as a definite integral: $\displaystyle \int_1^2 x\sqrt{x-1} \: dx$.}

\step{1}{As defined previously, $u = x - 1$. Change the limits of integration to $u$.}

When $x = 1$, $u = 1 - 1 = 0$.

When $x = 2$, $u = 2 - 1 = 1$.

\step{2}{Evaluate at the new limits of integration.}

\begin{gather*}
\left[ \frac{2}{5} u^{5/2} + \frac{2}{3} u^{3/2} \right]_{-1}^{0} \\
= \left(\frac{2}{5}+\frac{2}{3}\right) - (0+0) \\
= \boxed{\frac{16}{15}}
\end{gather*}

\subsection*{H18 Inverse trig integration}

\begin{gather*}
\int \frac{1}{\sqrt{a^2 - x^2}} \: dx = \arcsin \left( \frac{x}{a} \right) + C \\
\int \frac{1}{a^2 + x^2} \: dx = \frac{1}{a} \arctan \left( \frac{x}{a} \right) + C \\
\int \frac{1}{x \sqrt{x^2 - a^2}} \: dx = \frac{1}{a} \arcsec \left( \frac{|x|}{a} \right) + C
\end{gather*}

(no need to memorize formulas - memorize forms, use $u$-substitution instead)

\example{Evaluate $\displaystyle \int \frac{1}{\sqrt{16 - 9x^2}} \: dx$.}

\step{1}{Reduce the constant to 1.}

\begin{gather*}
\int \frac{1}{\sqrt{16 - 9x^2}} \: dx \\
= \int \frac{1}{16\sqrt{1 - \frac{9}{16}x^2}} \: dx \\
= \frac{1}{16} \int \frac{1}{\sqrt{1 - \left(\frac{3}{4}x\right)^2}} \: dx
\end{gather*}

\step{2}{Use $u$-substitution with $u = \frac{3}{4}x$.}

\begin{gather*}
u = \frac{3}{4}x \\
du = \frac{3}{4} dx \\
\frac{4}{3} du = dx \\
\frac{1}{16} \cdot \frac{4}{3} \int \frac{1}{\sqrt{1 - u^2}} \: du \\
= \frac{1}{12} \int \frac{1}{\sqrt{1 - u^2}} \: du
\end{gather*}

\step{3}{Use the inverse trig formula.}

\begin{gather*}
\frac{1}{12} \int \frac{1}{\sqrt{1 - u^2}} \: du = \frac{1}{12} \arcsin(u) + C \\
= \boxed{\frac{1}{12} \arcsin\left(\frac{3}{4}x\right) + C}
\end{gather*}

\subsection*{H19 Integration by division}

Use long division or synthetic division when the degree of the numerator is greater than or equal to the degree of the denominator.

\example{Evaluate $\displaystyle \int \frac{x^2 + 3x + 5}{x + 1} \, dx$.}

\step{1}{Use long division to divide the polynomials.}

$$x^2 + 3x + 5 = x + 2 + \frac{3}{x + 1}$$

\step{2}{Rewrite the integral and solve.}

\begin{gather*}
\int \frac{x^2 + 3x + 5}{x + 1} \, dx \\
= \int (x + 2 + \frac{3}{x + 1}) \, dx \\
= \boxed{\frac{x^2}{2} + 2x + 3 \ln|x + 1| + C} \\
\end{gather*}

\subsection*{How to know which integration method to use}

\begin{itemize}
    \item \textbf{Basic antiderivatives:} Check if the integral matches a basic antiderivative formula.
    \item \textbf{$u$-substitution:} If the integral contains a function and its derivative, consider $u$-substitution.
    \item \textbf{Change of variable:} Use when u-substitution is not sufficient and a more complex substitution is needed.
    \item \textbf{Long or synthetic division:} Use when the integrand is a rational function where the degree of the numerator is greater than or equal to the degree of the denominator.
    \item \textbf{$u$-substitution with trigonometry:} Use when the integrand contains inverse trig derivatives.
\end{itemize}
    
\end{document}