\documentclass[letterpaper, 12pt]{article}
\usepackage{graphicx} % Required for inserting images
\usepackage{textcomp}
\usepackage{fullpage}
\usepackage{amsmath}
\usepackage{xcolor}
\usepackage{float}
\usepackage{geometry}
\geometry{margin=1in}
\usepackage{enumitem}
\usepackage{hyperref}
\usepackage{microtype}
\usepackage{gensymb}
\usepackage{parskip}
\usepackage{tikz}
\usepackage{pgfplots}
\pgfplotsset{compat=1.18}
\usepackage{nicefrac}
\hypersetup{
    colorlinks=true,        % Enable colored links
    linkcolor=teal,         % Set color for internal links
    citecolor=teal,         % Set color for citations
    filecolor=teal,         % Set color for file links
    urlcolor=teal           % Set color for URLs
}

\DeclareMathOperator{\arcsec}{arcsec}

\newcommand{\example}[1]{\textcolor{blue}{\textbf{Example:} #1}}
\newcommand{\step}[2]{\textcolor{blue}{\textbf{Step #1:} #2}}

\begin{document}

\begin{center}
\textbf{{\Large AP Calculus BC Test 2}}
\end{center}

\subsection*{H10 Linearization}

Linearization is a method used to approximate the value of a function near a given point using the tangent line at that point. 

\example{Approximate the cube root of 82 using linearization.}

\step{1}{Identify the function and the point of tangency.}

$$f(x) = \sqrt[3]{x}, \quad a = 81$$

\step{2}{Find the derivative of the function.}

$$f'(x) = \frac{1}{3x^{2/3}}$$

\step{3}{Evaluate the function and its derivative at the point of tangency.}

$$f(81) = 3, \quad f'(81) = \frac{1}{27}$$

\step{4}{Write the equation of the tangent line.}

\subsection*{H11 Optimization}

\subsection*{H12 Estimating with Riemann sums}

\begin{description}
\item [RRAM] Right Rectangular Approximation Method - heights of rects = heights of right endpoints
\item [LRAM] Left Rectangular Approximation Method - heights of rects = heights of left endpoints
\item [MRAM] Midpoint Rectangular Approximation Method - heights of rects = heights of midpoints
\item [Trapezoidal approximation] Average of RRAM and LRAM
\end{description}



\subsection*{H13 Writing and interpreting Riemann sums}

\subsection*{H14 Riemann sums and definite integrals}

\subsection*{H15 Antiderivatives and indefinite integration}

\subsection*{H16 The Fundamental Theorem of Calculus}

\subsection*{H17 Integration by u-substitution and change of variable}

\subsection*{H18 Inverse trig integration}

\begin{gather*}
\int \frac{1}{\sqrt{a^2 - x^2}} \: dx = \arcsin \left( \frac{x}{a} \right) + C \\
\int \frac{1}{a^2 + x^2} \: dx = \frac{1}{a} \arctan \left( \frac{x}{a} \right) + C \\
\int \frac{1}{x \sqrt{x^2 - a^2}} \: dx = \frac{1}{a} \arcsec \left( \frac{|x|}{a} \right) + C
\end{gather*}

(no need to memorize formulas - memorize forms, use $u$-substitution instead)

\example{Evaluate $\displaystyle \int \frac{1}{\sqrt{16 - 9x^2}} \: dx$.}

\step{1}{Reduce the constant to 1.}

\begin{gather*}
\int \frac{1}{\sqrt{16 - 9x^2}} \: dx \\
= \int \frac{1}{16\sqrt{1 - \frac{9}{16}x^2}} \: dx \\
= \frac{1}{16} \int \frac{1}{\sqrt{1 - \left(\frac{3}{4}x\right)^2}} \: dx
\end{gather*}

\step{2}{Use $u$-substitution with $u = \frac{3}{4}x$.}

\begin{gather*}
u = \frac{3}{4}x \\
du = \frac{3}{4} dx \\
\frac{4}{3} du = dx \\
\frac{1}{16} \cdot \frac{4}{3} \int \frac{1}{\sqrt{1 - u^2}} \: du \\
= \frac{1}{12} \int \frac{1}{\sqrt{1 - u^2}} \: du
\end{gather*}

\step{3}{Use the inverse trig formula.}

\begin{gather*}
\frac{1}{12} \int \frac{1}{\sqrt{1 - u^2}} \: du = \frac{1}{12} \arcsin(u) + C \\
= \boxed{\frac{1}{12} \arcsin\left(\frac{3}{4}x\right) + C}
\end{gather*}

\subsection*{H19 Integration by division}

Use long division or synthetic division when the degree of the numerator is greater than or equal to the degree of the denominator.

\subsection*{How to know which integration method to use}

\begin{itemize}
    \item \textbf{Basic antiderivatives:} Check if the integral matches a basic antiderivative formula.
    \item \textbf{$u$-substitution:} If the integral contains a function and its derivative, consider $u$-substitution.
    \item \textbf{Change of variable:} Use when the 
    \item \textbf{Long or synthetic division:} Use when the integrand is a rational function where the degree of the numerator is greater than or equal to the degree of the denominator.
    \item \textbf{$u$-substitution with trigonometry:} Use when the integrand contains inverse trig derivatives.
\end{itemize}
    
\end{document}