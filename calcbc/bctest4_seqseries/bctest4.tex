\documentclass[letterpaper, 12pt]{article}
\usepackage{graphicx} % Required for inserting images
\usepackage{textcomp}
\usepackage{fullpage}
\usepackage{amsmath}
\usepackage{xcolor}
\usepackage{float}
\usepackage{enumitem}
\usepackage{geometry}
\geometry{margin=1in}
\usepackage{enumitem}
\usepackage{hyperref}
\usepackage{microtype}
\usepackage{gensymb}
\usepackage{parskip}
\usepackage{tikz}
\usepackage{pgfplots}
\pgfplotsset{compat=1.18}
\usepackage{nicefrac}
\hypersetup{
    colorlinks=true,        % Enable colored links
    linkcolor=teal,         % Set color for internal links
    citecolor=teal,         % Set color for citations
    filecolor=teal,         % Set color for file links
    urlcolor=teal           % Set color for URLs
}

\DeclareMathOperator{\arcsec}{arcsec}

\newcommand{\example}[1]{\textcolor{blue}{\textbf{Example:} #1}}
\newcommand{\step}[2]{\textcolor{blue}{\textbf{Step #1:} #2}}

\begin{document}

\begin{center}
\textbf{{\Large AP Calculus BC Test 4}}
\end{center}

\subsection*{H1 Sequences}

A \textbf{sequence} is a list of elements.

\begin{align*}
2,4,6,8\dots \quad & a_n = 2n \quad \text{arithmetic sequence} \\
1, \frac12, \frac14, \frac18, \frac{1}{16}\dots \quad & a_n = \frac{1}{2^{n-1}} \quad \text{geometric sequence}\\
1, \frac12, \frac16, \frac{1}{24}, \frac{1}{120}\dots \quad & a_n = \frac{1}{n!}
\end{align*}

\subsection*{H2 Series and convergence}

A \textbf{series} is the sum of the elements of a sequence.

\subsubsection*{Vocabulary and formulas}

\begin{align*}
\text{\textbf{Infinite series:}} & \quad \sum_{n=1}^\infty a_n = a_1 + a_2 + a_3 + \dots \\ 
\text{\textbf{Geometric series:}} & \quad \sum_{n=0}^\infty ar^n \\
\text{\textbf{Sum of geometric series:}} & \quad S = \frac{a}{1-r} \quad \text{if } |r| < 1 \\
\text{\textbf{Partial sum:}} & \quad S_n \\
\text{\textbf{Sum of series:}} & \quad S \\
\end{align*}

\subsubsection*{$n$th-term test}

\textcolor{red}{\textbf{If $\displaystyle \sum_{n=1}^\infty a_n$ \textit{converges}, then $\displaystyle \lim_{n \to \infty} a_n = 0.$}}

\textcolor{red}{This is \textit{not} saying that if the limit of the terms in the sequence goes to 0, then the series converges. If the limit goes to 0, further tests are needed to determine convergence.}

\textcolor{red}{\textbf{If $\displaystyle \lim_{n \to \infty} a_n \neq 0,$ then $\displaystyle \sum_{n=1}^\infty a_n$ \textit{diverges}.}}

\example{Determine whether the series $\displaystyle \sum_{n=1}^\infty \frac{n^2 + 2}{n}$ converges or diverges.}

Determine the limit:

$$\lim_{n \to \infty} \frac{n^2 + 2}{n} = \infty$$

because the degree of the numerator is greater than the degree of the denominator.

Since the limit does not equal 0, the series \textbf{diverges} by the $n$th-term test.

\subsubsection*{Geometric series test}

All that is needed to prove convergence of a geometric series is to show that the common ratio $r$ satisfies $|r| < 1$.

\example{Determine whether the series $\displaystyle \sum_{n=0}^\infty \left(\frac{3}{4}\right)^n$ converges or diverges. If it converges, find the sum.}

\step{1}{Find the common ratio and determine whether the series converges or diverges.}

The common ratio is $r = \displaystyle \frac{3}{4}$. Since $|r| = \displaystyle \frac{3}{4} < 1$, the series \textbf{converges}.

\step{2}{Find the sum.}

\begin{gather*}
S = \frac{a}{1-r} \\
= \frac{1}{1 - \frac{3}{4}} \\
= \frac{1}{\frac{1}{4}} \\
= \boxed{\text{converges to 4}}
\end{gather*}

\subsubsection*{Telescoping series test}

A \textbf{telescoping series} is a series where many terms cancel out when writing the partial sums. \underline{These often involve partial fractions.}

\example{Determine whether the series $\displaystyle \sum_{n=1}^\infty \frac{1}{n(n+1)}$ converges or diverges. If it converges, find the sum.}

\step{1}{Separate using partial fractions.}

\begin{gather*}
\frac{1}{n(n+1)} = \frac{A}{n} + \frac{B}{n+1} \\
A(n+1) + B(n) = 1 \\
An + A + Bn = 1 \\
A = 1 \\
B = -1 \\
\frac{1}{n(n+1)} = \frac{1}{n} - \frac{1}{n+1} \\
\end{gather*}

\step{2}{Write out the sequence.}

$$\sum_{n = 1}^\infty \frac{1}{n} - \frac{1}{n+1} = 1-\frac12+\frac12-\frac13+\frac13-\frac14+\frac14 \dots - \frac{1}{n+1}$$

\step{3}{Take the limit of the series.}

\begin{gather*}
S = 1 - \frac{1}{n+1} \\
\lim_{n \to \infty} 1 - \frac{1}{n+1}\\
= \boxed{\text{converges to 1}}
\end{gather*}

\subsection*{H3 Integral test and $p$-series}

\subsubsection*{Integral test}

If $f(x)$ is \textbf{positive, continuous, and decreasing} for $x \geq 1$ and $f(n) = a_n,$ then the series $\displaystyle \sum_{n=1}^\infty a_n$ and the integral $\displaystyle \int_1^\infty f(x) \, dx$ either both converge or both diverge.

\subsubsection*{$p$-series}

A \textbf{$p$-series} is a series of the form $\displaystyle \sum_{n=1}^\infty \frac{1}{n^p}$ where $p$ is a positive constant.

If $p > 1,$ the series \textit{converges}. Otherwise, the series \textit{diverges}.

A special case occurs at $p = 1,$ which is the harmonic series $\displaystyle \sum_{n=1}^\infty \frac{1}{n}$. This series \textit{diverges}.

\subsection*{H4 Comparison tests}

\subsubsection*{Direct comparison test}

The direct comparison test involves comparing the given series to a known series.

Given $a_n$ and $b_n$ are \textbf{positive terms} for all $n$:

\textbf{If $\displaystyle \sum b_n$ \textit{converges} and $a_n \leq b_n$ for all $n,$ then $\displaystyle \sum a_n$ \textit{converges}.}

\textbf{If $\displaystyle \sum a_n$ \textit{diverges} and $a_n \leq b_n$ for all $n,$ then $\displaystyle \sum b_n$ \textit{diverges}.}

\example{Determine whether the series $\displaystyle \sum_{n=1}^\infty \frac{n^2+2}{n^4+5}$ converges or diverges.}

\step{1}{Choose a series to compare to and set up the inequality.}

$$\frac{n^2+2}{n^4+5} < \frac{n^2+2}{n^4}$$

\step{2}{Simplify the inequality.}

$$\frac{n^2+2}{n^4} = \frac{1}{n^2} + \frac{2}{n^4}$$

\step{3}{Determine whether the comparison series converges or diverges.}

Because both $\displaystyle \sum_{n=1}^\infty \frac{1}{n^2}$ and $\displaystyle \sum_{n=1}^\infty \frac{2}{n^4}$ are $p$-series with $p > 1,$ both series \textbf{converge.} Therefore, \fbox{by the direct comparison test, the original series \textbf{converges.}}

\subsubsection*{Limit comparison test}

\textbf{Given $a_n$ and $b_n$ are positive terms for all $n,$ and $\displaystyle \lim_{n \to \infty} \frac{a_n}{b_n} = c,$ where $c$ is a finite number and $c > 0,$ \textbf{then $\displaystyle \sum a_n$ and $\displaystyle \sum b_n$ either both converge or both diverge.}}

\example{Determine whether the series $\displaystyle \sum_{n=0}^\infty \frac{1}{3^n-n}$ converges or diverges given that the series has positive terms.}

\step{1}{Choose a series to compare to.}

Because the dominant term on the denominator is $3^n,$ we can compare to the geometric series $\displaystyle \sum_{n=0}^\infty \frac{1}{3^n}.$

\step{2}{Find the limit of the ratio of the two series.}

\begin{gather*}
\lim_{n \to \infty} \frac{\frac{1}{3^n-n}}{\frac{1}{3^n}} \\
= \lim_{n \to \infty} \frac{3^n}{3^n - n} \\
= \lim_{n \to \infty} \frac{1}{1 - \frac{n}{3^n}} \\
= 1
\end{gather*}

The limit is a finite, positive value, so we move on with the limit comparison test.

\step{3}{Determine whether the comparison series converges or diverges.}

$$\frac{1}{3^n} = \left(\frac{1}{3}\right)^n$$

Because the common ratio $r$ of the geometric series is $\frac13$ ($|r| < 1$), the series \textbf{converges.} Therefore, \fbox{by the limit comparison test, the original series \textbf{converges.}}

\subsection*{H5 Alternating series test}

The \textbf{alternating series test} can be used on any alternating series of the form $\displaystyle \sum_{n=1}^\infty (-1)^{n-1} a_n$ or $\displaystyle \sum_{n=1}^\infty (-1)^n a_n,$ where $a_n > 0$ to test if it \textbf{converges.}

\textbf{The series converges if both of the following conditions are met:}

\begin{enumerate}[label=(\alph*)]
    \item $\displaystyle a_{n+1} \leq a_n$ for all $n$ (the terms are decreasing)
    \item $\displaystyle \lim_{n \to \infty} a_n = 0$
    
\end{enumerate}

If either condition is not met, the test is \textbf{inconclusive.}

\example{Determine if the series $\displaystyle \sum_{n=1}^\infty \frac{(-1)^{n+1}}{n}$ converges or diverges.}

\step{1}{Check if the limit of the terms goes to 0.}

Because the degree of the denominator is greater than the degree of the numerator,

\begin{gather*}
\lim_{n \to \infty} \frac{1}{n} = 0
\end{gather*}

\step{2}{Check if the terms decrease.}

Because $\displaystyle \frac{1}{n+1} < \frac{1}{n}$ for all $n,$ the terms decrease.

\step{3}{Conclude and justify.}

\fbox{By the alternating series test, the series converges.}

\subsubsection*{Alternating Series Remainder}

The \textbf{remainder} $R_N$ is the absolute difference between the sum of the series $S$ and the $N$th partial sum $S_N$.

From the textbook:

``If a convergent alternating series satisfies the condition $a_{n+1} \leq a_n$, then the absolute value of the remainder $R_N$ involved in approximating the sum $S$ by $S_N$ is less than (or equal to) the first neglected term. That is,

$$|S-S_N| = |R_N| \leq a_{N+1}.$$''

In simpler terms, for a convergent alternating series that is decreasing, the error made by stopping at the $N$th term is less than or equal to the absolute value of the next term.

For example, for the series $\displaystyle \sum_{n=1}^\infty \frac{(-1)^{n+1}}{n},$ if we approximate the sum by stopping at the 4th term (which is $S_4$), the error, $|S-S_4|$, is less than or equal to $\displaystyle \frac{1}{5}$ (which is $a_5$).

$$|S - S_4| \leq \frac{1}{5}$$

\example{Use the alternating series error bound to find an upper and lower bound for the sum of the series $\displaystyle \sum_{n=1}^{\infty} (-1)^n \frac{3}{3n^\frac{1}{3}+5}$ by approximating with the first four terms.}

\step{1}{Find the fourth partial sum $S_4$.}

$$S_4 = -0.048$$

\step{2}{Find the next term $a_5$.}

$$a_5 = \frac{3}{3(5)^{\frac{1}{3}} + 5} \approx 0.296$$

\step{3}{Find the upper and lower bounds. Lower bound = $S_4 - a_5$, upper bound = $S_4 + a_5$.}

\begin{gather*}
\text{Lower bound} = -0.048 - 0.296 = -0.344 \\
\text{Upper bound} = -0.048 + 0.296 = 0.248
\end{gather*}

$$\boxed{-0.344 \leq S \leq 0.248}$$

\subsubsection*{Absolute and conditional convergence}

A series $\displaystyle \sum_{n=1}^\infty a_n$ is said to be \textbf{absolutely convergent} if the series of absolute values $\displaystyle \sum_{n=1}^\infty |a_n|$ converges.

If a series $\displaystyle \sum_{n=1}^\infty a_n$ converges but the series of absolute values $\displaystyle \sum_{n=1}^\infty |a_n|$ diverges, then the series is said to be \textbf{conditionally convergent}.

\subsection*{H7 Ratio and root tests}

\subsubsection*{Ratio test}
Given a series $\displaystyle \sum_{n=1}^\infty a_n,$ let
$$L = \lim_{n \to \infty} \left| \frac{a_{n+1}}{a_n} \right|.$$
\begin{itemize}
    \item If $L < 1,$ the series \textbf{converges absolutely.}
    \item If $L > 1,$ the series \textbf{diverges.}
    \item If $L = 1,$ the test is \textbf{inconclusive.}
\end{itemize}

\subsubsection*{Root test}

Given a series $\displaystyle \sum_{n=1}^\infty a_n,$ let
$$L = \lim_{n \to \infty} \sqrt[n]{|a_n|}.$$
\begin{itemize}
    \item If $L < 1,$ the series \textbf{converges absolutely.}
    \item If $L > 1,$ the series \textbf{diverges.}
    \item If $L = 1,$ the test is \textbf{inconclusive.}
\end{itemize}

\newpage 

\subsection*{Review of tests}

\begin{table}[H]
\begin{tabular}{|p{6em}|p{8em}|p{20em}|p{6em}|}
\hline
\textbf{Test} & \textbf{Conditions} & \textbf{Procedure} & \textbf{C or D?} \\
\hline
$n$th-term & none & Find $\displaystyle \lim_{n \to \infty} a_n.$ If the limit is \textbf{not} 0, the series diverges.  & Divergence \\
\hline
geometric series & none & Find the common ratio $r.$ If $|r| < 1,$ converges. Otherwise, diverges. & Both \\
\hline
$p$-series & none & If $p > 1,$ converges. Otherwise, diverges. & Both \\
\hline
telescoping series & none & Cancel the middle terms and take the limit as $n$ goes to infinity of the first and last term. If the limit exists and is finite, the series converges. & Both \\
\hline
integral test & positive, continuous, and decreasing & Evaluate $ \int_1^\infty f(x) \, dx.$ If the integral converges, the series converges. Otherwise, it diverges. & Both \\
\hline
direct comparison & positive & Compare original series to a known series. Designate the smaller series as $a_n$ and the greater as $b_n$ ($a_n < b_n$). \textbf{If $a_n$ converges, the original series converges. If $b_n$ diverges, the original series diverges.} & Both \\
\hline
limit comparison & positive & Compare original series $a_n$ to a known series $b_n$. Find $\displaystyle \lim_{n \to \infty} \frac{a_n}{b_n} = c.$ If $c$ is finite and $c > 0,$ both series either converge or diverge. & Both \\
\hline
alternating series & positive, decreasing & Check that $\displaystyle \lim_{n \to \infty} a_n = 0$ to conclude the series converges & Convergence \\
\hline
ratio & none & Find $\displaystyle \lim_{n \to \infty} \left| \frac{a_{n+1}}{a_n} \right| = L.$ If $L < 1,$ converges. If $L > 1,$ diverges. If $L = 1,$ inconclusive. & Both \\
\hline
root & none & Find $\displaystyle \lim_{n \to \infty} \sqrt[n]{|a_n|} = L.$ If $L < 1,$ converges. If $L > 1,$ diverges. If $L = 1,$ inconclusive. & Both \\
\hline
\end{tabular}
\end{table}

\newpage

\subsection*{Taylor polynomials}

The Taylor series of a function $f(x)$ centered at $x = a$ is given by:

$$f(x) = \sum_{n=0}^\infty \frac{f^{(n)}(a)}{n!} (x-a)^n = f(a) + f'(a)(x-a) + \frac{f''(a)}{2!}(x-a)^2 + \frac{f'''(a)}{3!}(x-a)^3 + \dots$$

An $n$th-degree Taylor polynomial is given by:

$$P_n(x) = a_0 + a_1(x-a) + \frac{a_2}{2!}(x-a)^2 + \frac{a_3}{3!}(x-a)^3 + \dots + \frac{a_n}{n!}(x-a)^n$$

where

Similar to linearization, this series approximates the function $f(x)$ near the point $x = a$ with a polynomial function. 

\example{Find the 5th-order (5 derivatives) Taylor polynomial for $f(x) = e^{1-x}$ centered at $x = 0$.}

\step{1}{Find the derivatives of $f(x)$ and evaluate them at $x = 0$.}

\begin{gather*}
f(0) = e \\
f'(0) = -e \\
f''(0) = e \\
f'''(0) = -e \\
f^{(4)}(0) = e \\
f^{(5)}(0) = -e \\
\end{gather*}

\step{2}{Substitute into the Taylor polynomial formula.}

$$P_5(x) = e - ex + \frac{e}{2!} x^2 - \frac{e}{3!} x^3 + \frac{e}{4!} x^4 - \frac{e}{5!} x^5$$

\subsection*{H10 Lagrange error bound}

The \textbf{Lagrange error bound} gives an upper bound for the error when approximating a function $f(x)$ using its Taylor polynomial $P_n(x)$:

$$|R_n(x)| \leq \frac{M}{(n+1)!} |x-a|^{n+1}$$

where:

\begin{itemize}
    \item $R_n(x)$ is the remainder (error) after approximating $f(x)$ with the $n$th-degree Taylor polynomial $P_n(x)$.
    \item $M$ is the \textbf{absolute value} of the \textbf{greatest value} of $|f^{(n+1)}(z)|$ for some $z$ \textbf{between the centre and the $x$-value of the function} that is being approximated.
    \item $n$ is the degree of the Taylor polynomial.
    \item $a$ is the center of the Taylor series.
    \item $x$ is the point at which the function is approximated.
\end{itemize}

\example{Find an upper bound for the error when approximating $f(x) = \ln(x)$ using a 3rd-degree Taylor polynomial centered at $x = 1$ to approximate $f(1.5)$.}

\step{1}{Find the 4th derivative of $f(x)$.}

\begin{gather*}
f(x) = \ln(x) \\
f'(x) = \frac{1}{x} \\
f''(x) = -\frac{1}{x^2} \\
f'''(x) = \frac{2}{x^3} \\
f^{(4)}(x) = -\frac{6}{x^4} \\
\end{gather*}

\step{2}{ Find $M$.}

The maximum absolute value of the derivative between $x = 1$ and $x = 1.5$ is at $x = 1$:

$$M = \left| f^{(4)}(1) \right| = 6$$

\step{3}{Substitute into the Lagrange error bound formula.}

\begin{gather*}
|R_3(1.5)| \leq \frac{6}{4!} |1.5 - 1|^4 \\
= \frac{6}{24} (0.5)^4 \\
= \boxed{0.003125}
\end{gather*}

\subsection*{H11 Power series}

The power series is a series of the form:

$$\sum_{n=0}^\infty a_n (x-c)^n = a_0 + a_1 (x-c) + a_2 (x-c)^2 + a_3 (x-c)^3 + \dots$$

where $a_n$ represents the coefficients of the series, $c$ is the center of the series, and $x$ is the variable.

\subsubsection*{Convergence of power series}

\textbf{Three cases:}

\begin{enumerate}
\item series converges only at $c$
\item series converges absolutely for all $x$
\item series converges for $|x-c| < R$ and diverges for $|x-c| > R$ (where $R$ is the radius of convergence); \textbf{interval of convergence} is the value of $x$ for which the series converges
\end{enumerate}

To find the radius and interval of convergence, use the \textbf{ratio test} or the \textbf{root test.}

\textbf{Important!}

\begin{enumerate}
\item \textbf{Check endpoints} to be sure whether the constraints should be strict or include endpoints. (for example, $-1 < x < 1$ or $-1 \leq x < 1$)
\item The radius of convergence is \textbf{the center minus the endpoint} and is always non-negative. (for example, if the interval of convergence is $-2 < x < 4,$ the center is 1 and the radius is 3)
\end{enumerate}

\example{Find the center of the power series $\displaystyle \sum_{n=0}^\infty \frac{x^n}{n \sqrt{n} \cdot 3^n}$, the radius of convergence, and the interval of convergence.}

\step{1}{Find the center.}

The series is in the form $\displaystyle \sum_{n=0}^\infty a_n (x-c)^n.$ So, $\boxed{c = 0}.$

\step{2}{Use the ratio test to find the radius of convergence.}

\begin{gather*}
L = \lim_{n \to \infty} \left| \frac{x^{n+1}}{(n+1) \sqrt{n+1} \cdot 3^{n+1}} \cdot \frac{n \sqrt{n} \cdot 3^n}{x^n} \right| \\
= \lim_{n \to \infty} \left| \frac{x}{3} \cdot \frac{n \sqrt{n}}{(n+1) \sqrt{n+1}} \right| \\
= \left| \frac{x}{3} \right| \lim_{n \to \infty} \frac{n}{n+1} \cdot \lim_{n \to \infty} \frac{\sqrt{n}}{\sqrt{n+1}} \\
= \frac{x}{3}
\end{gather*}

For convergence, $L < 1$:

\begin{gather*}
\left| \frac{x}{3} \right| < 1 \\
|x| < 3
\end{gather*}

So, the radius of convergence is \boxed{R = 3}.

\step{3}{Find the interval of convergence.}

\begin{gather*}
-1 < \frac{x}{3} < 1 \\
\boxed{-3 < x < 3}
\end{gather*}

\textbf{For any power series, the interval of convergence is the same for its derivative and its integral. However, the endpoints may differ.}

\example{Find the interval of convergence of $\displaystyle \sum_{n=0}^\infty \frac{(-1)^{n+1}(x-1)^{n+1}}{n+1}$, of its derivative, and its integral.}

\step{1}{Identify the center and interval of convergence of the original function.}

The center is $c = 1$, as 1 is subtracted from $x$.

Because this is a geometric series with common ratio $r = x-1$, the interval of convergence is:

\begin{gather*}
|x-1| < 1 \\
-1 < x-1 < 1 \\
\boxed{0 < x < 2}
\end{gather*}

\step{2}{Take the derivative.}

Treating $n$ (and $(-1)^{n+1}$)as a constant, take the derivative with respect to $x$ by using the power rule:

$$ \frac{d}{dx} \displaystyle \sum_{n=0}^\infty \frac{(-1)^{n+1}(x-1)^{n+1}}{n+1} = \displaystyle \sum_{n=0}^\infty (-1)^{n+1}(x-1)^n$$

\step{3}{Test the endpoints of the derivative.}

For $x=0$:

$$\displaystyle \sum_{n=0}^\infty (-1)^{n+1}(-1)^n$$

This series \textbf{diverges} because the limit of the terms does not go to 0 ($n$th-term test).

For $x=2$:

$$\displaystyle \sum_{n=0}^\infty (-1)^{n+1}(1)^n$$

This series \textbf{diverges} because the limit of the terms does not go to 0 ($n$th-term test).

$\boxed{ 0 < x < 2}$

\step{4}{Take the integral of the original function.}

$$\int \displaystyle \sum_{n=0}^\infty \frac{(-1)^{n+1}(x-1)^{n+1}}{n+1} \, dx = \displaystyle \sum_{n=0}^\infty \frac{(-1)^{n+1}(x-1)^{n+2}}{(n+1)(n+2)} + C$$

\step{5}{Test the endpoints of the integral.}

For $x=0$:

$$\displaystyle \sum_{n=0}^\infty \frac{(-1)^{n+1}(-1)^{n+2}}{(n+1)(n+2)} + C$$

This series \textbf{converges} by the alternating series test.

For $x=2$:

$$\displaystyle \sum_{n=0}^\infty \frac{(-1)^{n+1}(1)^{n+2}}{(n+1)(n+2)} + C$$

This series \textbf{converges} by the alternating series test.

$$\boxed{ 0 \leq x \leq 2}$$

\subsection*{H12 Representing functions as power series}

To represent a function as a power series, recall the equation

$$S = \frac{a}{1-r}$$

Using this function, identify $a$ and $r$, and then substitute into the formula to find the power series representation of the function.

\example{Find a power series representation for $\displaystyle g(x) = \frac{4x}{x^2+2x-3}$ and find the interval of convergence.}

\step{1}{Factor the denominator.} By applying partial fractions, we can rewrite the function as:

\begin{gather*}
4x = \frac{A}{x+3} + \frac{B}{x-1} \\
A(x-1) + B(x+3) = 4x \\
Ax - A + Bx + 3B = 4x \\
A + B = 4 \\
- A + 3B = 0 \\
A = 3 \\
B = 1 \\
g(x) = \frac{3}{x+3} + \frac{1}{x-1}
\end{gather*}

\step{2}{Identify $a$ and $r$ for each term.}

For $\displaystyle \frac{3}{x+3}$:

$$\frac{3}{x+3} = \frac{3/3}{x/3+3/3} = \frac{1}{1+x/3}$$

$a$ is 1 and $r$ is $-x/3$.

For $\displaystyle \frac{1}{x-1}$:

$$\frac{1}{x-1} = \frac{-1}{1-x}$$

$a$ is $-1$ and $r$ is $x$.

\step{3}{Substitute into the formula $S = \frac{a}{1-r}$ to find the power series representation.}

$$\boxed{\sum_{n=0}^{\infty} \left(\frac{-x}{3}\right)^n - \sum_{n=0}^{\infty} x^n}$$

\step{4}{Find the interval of convergence.}

For $\displaystyle \frac{3}{x+3}$:

\begin{gather*}
\left| \frac{-x}{3} \right| < 1 \\
|x| < 3 \\
-3 < x < 3
\end{gather*}

For $\displaystyle \frac{1}{x-1}$:

\begin{gather*}
|x| < 1 \\
-1 < x < 1
\end{gather*}

The interval of convergence is the intersection of the two intervals, which is \boxed{-1 < x < 1}.

\subsection*{H13 Taylor and Maclaurin series}

The three series to memorize are:

$$e^x = \sum_{n=0}^\infty \frac{x^n}{n!} = 1 + x + \frac{x^2}{2!} + \frac{x^3}{3!} + \dots$$
$$\sin x = \sum_{n=0}^\infty \frac{(-1)^n x^{2n+1}}{(2n+1)!} = x - \frac{x^3}{3!} + \frac{x^5}{5!} - \dots$$
$$\cos x = \sum_{n=0}^\infty \frac{(-1)^n x^{2n}}{(2n)!} = 1 - \frac{x^2}{2!} + \frac{x^4}{4!} - \dots$$

\example{Find the Maclaurin series for $\displaystyle \int (e^{-t/2}-1) dt$.}

\step{1}{Use the Maclaurin series for $e^x$ to find the Maclaurin series for $e^{-t/2}$.}

$$e^{-t/2} = \sum_{n=0}^\infty \frac{(-t/2)^n}{n!} = \sum_{n=0}^\infty \frac{(-1)^n}{n!} t^{2n}$$

\step{2}{Subtract 1 from the series.}

Because the first term of the series is 1, to subtract 1, start the series at $n=1$ instead of $n=0$:

$$e^{-t/2} - 1 = \sum_{n=1}^\infty \frac{(-1)^n}{n!} t^{2n}$$

\step{3}{Substitute the integrand for the series.}

$$\int (e^{-t/2}-1) dt = \int \sum_{n=1}^\infty \frac{(-1)^n}{n!} t^{2n} dt$$

\step{4}{Integrate the series.}

Because $\displaystyle \frac{(-1)^n}{n!}$ is a constant, it gets pulled out of the integral. The rest of the integral becomes

$$\int t^{2n} dt = \frac{t^{2n+1}}{2n+1} + C$$

Evaluated from 0 to $x$, the integral becomes

$$\frac{x^{2n+1}}{2n+1}$$

\step{5}{Substitute the integrated series back into the summation.}

$$\boxed{\sum_{n=1}^\infty \frac{(-1)^n}{n!} \cdot \frac{x^{2n+1}}{2n+1}}$$

\end{document}