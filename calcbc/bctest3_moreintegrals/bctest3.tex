\documentclass[letterpaper, 12pt]{article}
\usepackage{graphicx} % Required for inserting images
\usepackage{textcomp}
\usepackage{fullpage}
\usepackage{amsmath}
\usepackage{xcolor}
\usepackage{float}
\usepackage{enumitem}
\usepackage{geometry}
\geometry{margin=1in}
\usepackage{enumitem}
\usepackage{hyperref}
\usepackage{microtype}
\usepackage{gensymb}
\usepackage{parskip}
\usepackage{tikz}
\usepackage{pgfplots}
\pgfplotsset{compat=1.18}
\usepackage{nicefrac}
\hypersetup{
    colorlinks=true,        % Enable colored links
    linkcolor=teal,         % Set color for internal links
    citecolor=teal,         % Set color for citations
    filecolor=teal,         % Set color for file links
    urlcolor=teal           % Set color for URLs
}

\DeclareMathOperator{\arcsec}{arcsec}

\newcommand{\example}[1]{\textcolor{blue}{\textbf{Example:} #1}}
\newcommand{\step}[2]{\textcolor{blue}{\textbf{Step #1:} #2}}

\begin{document}

\begin{center}
\textbf{{\Large AP Calculus BC Test 3}}
\end{center}

\subsection*{H20 Slope fields and Euler's method}

A slope field is a graphical representation of a differential equation that shows the slope of the solution curve at each point in the plane.

For example, the slope field for the differential equation $\displaystyle \frac{dy}{dx} = x^2 - x -2$ can be drawn by calculating the slope at various points $(x, y)$ and drawing small line segments with those slopes. The result is (from Wikipedia):

\begin{figure}[H]
\centering
\includegraphics[width=0.5\textwidth]{slopefield.png}
\end{figure}

Euler's method is a technique used to approximate solutions to differential equations with a given initial value by using the slope of the function at a given point to estimate the value of the function at the next point.

\example{Use a table and Euler's method to approximate the value of $y$ at $x=1$ for the differential equation $\displaystyle \frac{dy}{dx} = x + y$ with the initial condition $y(0) = 1$, using a step size of $h=0.5$. Use $\Delta y = \displaystyle \frac{dy}{dx} \cdot \Delta x$ to find the change in $y$ at each step.}

\step{1}{Make a table with the initial condition and known values.}

\begin{table}[H]
\centering
\begin{tabular}{|c|c|c|c|c|c|}
\hline
$x$ & $y$ & $\Delta x$ & $\frac{dy}{dx} = x + y$ & $\Delta y$ & $(x + \Delta x, y + \Delta y)$ \\
\hline
0 & 1 & 0.5 & - & - & $(0.5, y)$ \\
0.5 & - & 0.5 & - & - & $(1, y)$ \\
\hline
\end{tabular}
\end{table}

\step{2}{Calculate the slope $\displaystyle \frac{dy}{dx}$ at each point and use it to find $\Delta y$.}

\begin{table}[H]
\centering
\begin{tabular}{|c|c|c|c|c|c|}
\hline
$x$ & $y$ & $\Delta x$ & $\frac{dy}{dx} = x + y$ & $\Delta y$ & $(x + \Delta x, y + \Delta y)$ \\
\hline
0 & 1 & 0.5 & 1 & 0.5 & $(0.5, 1.5)$ \\
0.5 & 1.5 & 0.5 & 2 & 1 & $(1, 2.5)$ \\
\hline
\end{tabular}
\end{table}

\example{Use tangent lines to do the same problem.}

\step{1}{Find the equation of the tangent line at the initial point.}

Given:

\begin{gather*}
\frac{dy}{dx} = x + y \\
y(0) = 1 \\
\Delta x = 0.5
\end{gather*}

Finding the tangent line:

\begin{gather*}
\frac{dy}{dx} = 0 + 1 = 1 \\
y - 1 = 1(x - 0) \\
y = x + 1
\end{gather*}

\step{2}{Use the tangent line to approximate $y$ at $x=0.5$.}

At $x=0.5$, $y = 0.5 + 1 = 1.5$.

\step{3}{Use the new point to find the next tangent line.}

\begin{gather*}
\frac{dy}{dx} = 1.5 + 0.5 = 2 \\
y - 1.5 = 2(x-0.5) \\
y = 2x + 0.5
\end{gather*}

\step{4}{Use the new tangent line to approximate $y$ at $x=1$.}

At $x=1$, $y = 2(1) + 0.5 = \boxed{2.5}$.

Both methods yield the same approximation of $y(1) \approx 2.5$.

\subsection*{H21 Separable differential equations}

A separable differential equation is one that can be expressed in the form $\frac{dy}{dx} = f(x)g(y)$, allowing the variables to be separated on opposite sides of the equation for integration.

\example{Solve the separable differential equation $\displaystyle \frac{dy}{dx} = \frac{x}{y}$ with the initial condition $y(0) = 2$.}

\step{1}{Separate the variables.}

\begin{gather*}
\frac{dy}{dx} = \frac{x}{y} \\
y \: dy = x \: dx
\end{gather*}

\step{2}{Integrate both sides to find the general solution.}

\begin{gather*}
\int y \: dy = \int x \: dx \\
\frac{1}{2} y^2 = \frac{1}{2} x^2 + C \\
y^2 = x^2 + C \\
\end{gather*}

\step{3}{Use the initial condition to find the particular solution.}

\begin{gather*}
y^2 = x^2 + C\\
y(0) = 2 \\
2^2 = 0^2 + C \\
C = 4 \\
y = \pm \sqrt{x^2+4} \\
2 = \pm \sqrt{0^2 + 4} \\
2 = \sqrt{4} \\
\boxed{y = \sqrt{x^2 + 4}}
\end{gather*}

\subsection*{H22 Logistic equations}

A logistic equation is a type of differential equation that models population growth with a carrying capacity.

Typically, the logistic equation is expressed as $$\frac{dP}{dt} = kP(M-P)$$

where $P$ is the population size, $k$ is the growth rate, and $M$ is the carrying capacity.

Important things to note:
\begin{itemize}
\item The population grows fastest at $P = \displaystyle \frac{M}{2}$, or in the middle of the curve at the point of inflection.
\item As $P$ approaches $M$, the growth rate slows down and the population stabilizes. The rate of change approaches zero as the population approaches its carrying capacity.
\item If the equation is not in the standard form, it may need to be manipulated algebraically to identify $k$ and $M$.
\end{itemize}

\subsection*{H23 (review of integrals, skipped)}

\subsection*{H24 Integration by parts, solving for the integral, and tabular integration}

\subsubsection*{Integration by parts}

Integration by parts is a technique used to integrate products of functions following the formula\footnote{This formula can be memorized with the mnemonic device ``ultraviolet voodoo'', origin unknown :)}:

$$\int u \: dv = uv - \int v \: du$$

where $u$ and $dv$ are parts of the original integral.

To identify $u$, use the initialism \textbf{LIATE}, which stands for \textbf{L}ogarithmic, \textbf{I}nverse trigonometric, \textbf{A}lgebraic, \textbf{T}rigonometric, and \textbf{E}xponential functions. The function that appears first in this list should be chosen as $u$.

\example{Use integration by parts to evaluate the integral $\displaystyle \int x e^x \: dx$.}

\step{1}{Identify $u$ and $dv$ with $du$ and $v$.}

Let $u = x$ (algebraic) and $dv = e^x \: dx$. Thus, $du = dx$ and $v = e^x$.\footnote{While $u=x$ is not a valid choice for u-substitution since it does not simplify the integral, it does work well for integration by parts.}

\step{2}{Follow the formula.}

\begin{gather*}
uv - \int v \: du \\
= x e^x - \int e^x \: dx
\end{gather*}

\step{3}{Integrate.}

\begin{gather*}
x e^x - \int e^x \: dx \\ 
= \boxed{x e^x - e^x + C} \\
\end{gather*}

\subsubsection*{Solving for the integral}

Sometimes, applying integration by parts results in an equation that contains the original integral. In such cases, solve for the integral algebraically.

Consider the following integral:

$$\int e^x \cos x \: dx$$

This integral looks solvable, but applying integration by parts twice will lead back to the original integral. To test this, let $u = e^x$ and $dv = \cos x \: dx$, and therefore $du = e^x \: dx$ and $v = \sin x$. Following the formula 

$$ \int u \: dv = uv - \int v \: du $$

yields:

$$ \int e^x \cos x \: dx = e^x \sin x - \int e^x \sin x \: dx $$

Now, apply integration by parts again to the remaining integral $\int e^x \sin x \: dx$. Let $u = e^x$ and $dv = \sin x \: dx$, so $du = e^x \: dx$ and $v = -\cos x$. Applying the formula again gives:

$$ \int e^x \sin x \: dx = -e^x \cos x + \int e^x \cos x \: dx $$

Substituting this back into the previous equation results in:

$$ \int e^x \cos x \: dx = e^x \sin x - \left( -e^x \cos x + \int e^x \cos x \: dx \right) $$

This simplifies to:

$$ \int e^x \cos x \: dx = e^x \sin x + e^x \cos x - \int e^x \cos x \: dx $$

Now, the original integral is on both sides of the equation. To solve for it, add $\int e^x \cos x \: dx$ to both sides:

$$ 2 \int e^x \cos x \: dx = e^x (\sin x + \cos x) $$

Finally, divide both sides by 2 to isolate the integral:

$$ \int e^x \cos x \: dx = \boxed{\frac{e^x (\sin x + \cos x)}{2} + C} $$

\subsubsection*{Tabular integration}

Tabular integration is a method used to simplify the process of integration by parts when one function can be differentiated repeatedly until it becomes zero, and the other function can be integrated repeatedly.

\example{Use tabular integration to evaluate the integral $\displaystyle \int x^3 e^x \: dx$.}

\step{1}{Set up the table and differentiate $u$ until the left side reaches 0. Integrate $dv$ the same number of times.}

\begin{table}[H]
\centering
\begin{tabular}{|c|c|}
\hline
$u$ & $dv$ \\
\hline
$x^3$ & $e^x$ \\
$3x^2$ & $e^x$ \\
$6x$ & $e^x$ \\
$6$ & $e^x$ \\
$0$ & $e^x$ \\
\hline
\end{tabular}
\end{table}

\step{2}{Multiply diagonally, alternating signs, and sum the results.}

\begin{gather*}
+ x^3 e^x \\
- 3x^2 e^x \\
+ 6x e^x \\
- 6 e^x \\
\end{gather*}

\step{3}{Write the final answer.}

$$\int x^3 e^x \: dx = \boxed{e^x (x^3 - 3x^2 + 6x - 6) + C}$$

\end{document}