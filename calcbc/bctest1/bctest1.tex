\documentclass[letterpaper, 12pt]{article}
\usepackage{graphicx} % Required for inserting images
\usepackage{textcomp}
\usepackage{fullpage}
\usepackage{amsmath}
\usepackage{xcolor}
\usepackage{float}
\usepackage{geometry}
\geometry{margin=1in}
\usepackage{enumitem}
\usepackage{hyperref}
\usepackage{microtype}
\usepackage{gensymb}
\usepackage{parskip}
\usepackage{tikz}
\usepackage{pgfplots}
\pgfplotsset{compat=1.18}
\usepackage{nicefrac}
\hypersetup{
    colorlinks=true,        % Enable colored links
    linkcolor=teal,         % Set color for internal links
    citecolor=teal,         % Set color for citations
    filecolor=teal,         % Set color for file links
    urlcolor=teal           % Set color for URLs
}

\newcommand{\example}[1]{\textcolor{blue}{\textbf{Example:} #1}}
\newcommand{\step}[2]{\textcolor{blue}{\textbf{Step #1:} #2}}

\begin{document}

\begin{center}
\textbf{{\Large AP Calculus BC Test 1}}
\end{center}

\subsection*{H1 Rates of change}

The \textbf{average rate of change} (also AROC, slope of the secant line) of a function $f$ over the interval $[a, b]$ is given by:

\[\frac{f(b) - f(a)}{b - a}\]

The \textbf{instantaneous rate of change} (also IROC, slope of the tangent line) of a function $f$ at a point $x = c$ is given by the derivative:

\[f'(c) = \lim_{h \to 0} \frac{f(c + h) - f(c)}{h}\]

This is also known as the \textbf{difference quotient} or the \textbf{limit definition of the derivative.}

\example{A tangent line to the graph $f(x)$ at the point $x = a$ uses the instantaneous rate of change for the slope. Write the equation of the tangent line to $f(x) = -x^2+5$ and then use the tangent line to approximate $f(5.1)$.}

\step{1}{Find the derivative by the power rule.}

\begin{gather*}
f(x) = -x^2+5 \\
f'(x) = -2x
\end{gather*}

\step{2}{Find the equation of the tangent line at x = 5.}

\begin{gather*}
f'(5) = -10 \\
f(5) = -5^2 + 5 = -20 \\
y+20 = -10(x-5) \\
y = -10x + 50 - 20 \\
y = -10x + 30
\end{gather*}

\step{3}{Find $f(5.1)$ using the tangent line.}

\begin{gather*}
f(5.1) \approx -10(5.1) + 30 \\ 
= -51 + 30 \\
= \boxed{-21}
\end{gather*}

\subsection*{H2 Limits and continuity}

The \textbf{limit} of a function is the y-value the graph approaches. Limit values are not affected by holes, random points, or sharp turns. However, there are some conditions for a limit to exist:

\begin{enumerate}
\item \textbf{The limit from the left must equal the limit from the right.} In math terms:
\[\lim_{x \to c^-} f(x) = \lim_{x \to c^+} f(x)\]
If the limit from the left does not equal the limit from the right, then the limit does not exist (DNE). \textit{If the limit from the left equals the limit from the right and the value of the function equals the value of the limit, then the function is \textbf{continuous} at the given point.}
\item \textbf{The limit must approach a finite value.} If the limit approaches infinity, then the limit does not exist (DNE). However, if the limit approaches $\infty$ or $-\infty$ from both the left and the right, the limit can be considered to exist as either $\infty$ or $-\infty$.
\end{enumerate}

There are different \textbf{types of discontinuities} to consider:

\paragraph{Removable} Removable discontinuities can be ``removed'' when the function is defined at the point of discontinuity. This is represented by a hole in the graph. The limit exists at the hole.

\paragraph{Non-removable jump} Jump discontinuities occur when the limit from the left does not equal the limit from the right and both limits approach finite values. This is represented by a jump in the graph. The limit does not exist at the jump.

\paragraph{Asymptotic} Asymptotic discontinuities occur when the limit from the left or right approaches $\infty$ or $-\infty$. This is represented by a vertical asymptote in the graph. The limit does not exist at the asymptote or can be regarded as $\infty$ or $-\infty$ if the asymptote is an even asymptote.

\paragraph{Oscillating} Oscillating discontinuities occur when the function oscillates between two values as it approaches a point. This is represented by a wavy line in the graph that goes up and down indefinitely. The limit does not exist at the oscillation.

\example{In the graph of $\displaystyle f(x) = \frac{x+3}{x^2-9}$, at what $x$-values do removable discontinuities occur? Asymptotic discontinuities? What is the limit ($y$-value) as $x$ approaches the removable discontinuity?}

\step{1}{Factor the function and cross out the common factors.}

\begin{gather*}
f(x) = \frac{x+3}{(x-3)(x+3)} \\
f(x) = \frac{1}{x-3}, x \neq -3
\end{gather*}

The removable discontinuity occurs at $ x = -3 $.

\step{2}{Find the limit as $x$ approaches the removable discontinuity.}

\begin{gather*}
\lim_{x \to -3} f(x)
= \lim_{x \to -3} \frac{1}{x-3} \\
= \frac{1}{-3-3} \\
= \boxed{-\frac{1}{6}}
\end{gather*}

\step{3}{Find the asymptotic discontinuity.}

The asymptotic discontinuity occurs at $ x = 3 $ because $x$ cannot equal 3 under an y circumstance.

Check with \href{https://www.desmos.com/calculator}{Desmos}.

\subsubsection*{What to do to find...}

\paragraph{Limits at a point} Direct substitution or algebraic manipulation (factoring, rationalizing, etc.).

\paragraph{Limits at infinity} Consider the dominant term in both the numerator and denominator, then simplify.

\paragraph{Horizontal asymptotes} Find the limit as $x$ approaches $\infty$ and $-\infty$. If the limit approaches a finite value, then that value is the horizontal asymptote.

\paragraph{Indeterminate forms} Use L'Hopital's rule and find the derivative of the top over the derivative of the bottom (separately, not with quotient rule). DO NOT WRITE ``$f(x) = \frac{0}{0}$''!!!! (``$\frac{0}{0}$ indeterminate form'' is fine.)

\subsection*{H3 Non-traditional limits}

To find non-traditional limits, apply one of the following:

\begin{itemize}
\item Evaluate the limit separately from the left and from the right and see if they match.
\item If the limit is a composition, evaluate the limit of the inner function first, then evaluate the limit of the outer function as it approaches the limit of the inner function.
\item If the limit is a composition and the outer function has a jump discontinuity, evaluate whether the inner function approaches the value of the jump discontinuity from the top (right, +) or from the bottom (left, -), then evaluate the outer function as a one-sided limit.
\item If the limit is a composition and the inner function does not merely approach a value but rather stays consistently at that value, then evaluate the outer function at the value of the limit of the inner function.
\end{itemize}

\subsection*{H4 Differentiation}

Refer to \href{https://www.mathsisfun.com/calculus/derivatives-rules.html}{derivative rules} for a full list.

\subsection*{H5 Graphs of derivatives and derivatives of inverse (trig) functions}

The point $(a, b)$ is on $f(x)$, and $(b, a)$ is on $f^{-1}(x)$, its inverse. The derivative of the inverse function is given by:

\[(f^{-1})'(b) = \frac{1}{f'(a)}\]

In other words, the derivative of the inverse at the y-value of a function equals the reciprocal of the derivative of the function at the x-value.

\begin{gather*}
\frac{d}{dx} (\sin^{-1} x) = \frac{1}{\sqrt{1-x^2}} \\
\frac{d}{dx} (\cos^{-1} x) = \frac{-1}{\sqrt{1-x^2}} \\
\frac{d}{dx} (\tan^{-1} x) = \frac{1}{1+x^2}
\end{gather*}

\href{https://english.mathe-online.at/tests/diff1/ablerkennen.html}{The big derivative puzzle}

\subsection*{H6 Position, acceleration, velocity}

\textbf{Velocity} is the first derivative of \textbf{position} (position is the integral of velocity) \\
\textbf{Acceleration} is the first derivative of velocity (velocity is the integral of acceleration) \\
Acceleration is the second derivative of position \\

\textbf{Distance} vs \textbf{displacement}: Distance is the total length traveled, while displacement is the change in position (final position - initial position). To calculate total distance, find the critical points of the velocity function, then evaluate the position function at those points and add the absolute values of the differences.

\subsection*{H7 Implicit differentiation}

\example{Find $\frac{dy}{dx}$ if $x^2+xy+y^2 = 7$.}

\step{1}{Differentiate both sides with respect to $x$.}

\begin{gather*}
\frac{d}{dx} (x^2+xy+y^2) = \frac{d}{dx} (7) \\
2x + x \frac{dy}{dx} + y + 2y \frac{dy}{dx} = 0
\end{gather*}

\step{2}{Solve.}
\begin{gather*}
x \frac{dy}{dx} + 2y \frac{dy}{dx} = -2x - y \\
\frac{dy}{dx} (x + 2y) = -2x - y \\
\boxed{\frac{dy}{dx} = \frac{-2x - y}{x + 2y}}
\end{gather*}

\subsection*{H8 Related rates}

\example{A 15-foot ladder leans against a vertical wall. The bottom is sliding away from the wall at a rate of 2 ft s$^{-1}$. Let $\theta$ be the angle between the ground and the ladder. At what rate is $\theta$ changing when the bottom of the ladder is 9 feet from the wall?}

\step{1}{Draw a picture, label the variables, then list what is given and what is to be found.}

\begin{itemize}
\item Given: $\frac{dx}{dt} = 2$ ft/s, $x = 9$ ft, ladder length = 15 ft
\item Find: $\frac{d\theta}{dt}$ when $x = 9$ ft
\end{itemize}

\step{2}{Write an equation relating the variables.}
\[\cos \theta = \frac{x}{15}\]

\step{3}{Differentiate implicitly with respect to $t$.}
\[-\sin \theta \frac{d\theta}{dt} = \frac{1}{15} \frac{dx}{dt}\]

\step{4}{Solve for $\frac{d\theta}{dt}$.}
\[\frac{d\theta}{dt} = \frac{-1}{15 \sin \theta} \frac{dx}{dt}\]
\step{5}{Find $\sin \theta$ when $x = 9$ ft.}
\begin{gather*}
\sin \theta = \frac{\text{opposite}}{\text{hypotenuse}} \\ = \frac{\sqrt{15^2 - 9^2}}{15} \\ = \frac{12}{15} \\ = \boxed{\frac{4}{5}}
\end{gather*}

\step{6}{Substitute known values and calculate.}
\begin{gather*}
\frac{d\theta}{dt} = \frac{-1}{15 \cdot \frac{4}{5}} \cdot 2 \\ = \boxed{\frac{-1}{6} \text{ rad/s}}
\end{gather*}

\subsection*{H9 Extrema and concavity}

\textbf{Critical points} occur when $f'(x) = 0$ or $f'(x)$ is undefined. To find critical points, find the derivative, set it equal to zero, and solve.

\textbf{Absolute extrema} are the highest and lowest points on a closed interval. To find absolute extrema, evaluate the function at the critical points and at the endpoints of the interval, then compare the values.

\textbf{Local extrema} are the highest and lowest points in a small neighborhood. To find local extrema, use the first or second derivative test.

\textbf{First derivative test}: Find the critical points, then test values in the intervals between the critical points to see if the derivative is positive or negative. If $f'(x)$ changes from positive to negative at a critical point, then $f(x)$ has a local maximum at that point. If $f'(x)$ changes from negative to positive at a critical point, then $f(x)$ has a local minimum at that point. If $f'(x)$ does not change sign at a critical point, then $f(x)$ has no local extremum at that point.

\textbf{Second derivative test}: Find the critical points, then find the second derivative. If $f''(x) > 0$ at a critical point, then $f(x)$ has a local minimum at that point. If $f''(x) < 0$ at a critical point, then $f(x)$ has a local maximum at that point. If $f''(x) = 0$ at a critical point, then the test is inconclusive.

\textbf{Concavity} describes the direction the graph is curving. If $f''(x) > 0$, then the graph is concave up. If $f''(x) < 0$, then the graph is concave down. \textbf{Inflection points} occur when $f''(x) = 0$ or $f''(x)$ is undefined. 



\end{document}